%----------------------------------------------------------------------------------------
%   PACKAGES AND OTHER DOCUMENT CONFIGURATIONS
%----------------------------------------------------------------------------------------

\documentclass[9pt]{resume} % Default font size, values from 8-12pt are recommended
\usepackage{multicol}
\setlength{\columnsep}{0mm}
%----------------------------------------------------------------------------------------
\usepackage{lipsum}


\begin{document}

%----------------------------------------------------------------------------------------
%   URLs
%----------------------------------------------------------------------------------------

\newcommand{\contactgithub}{github.com/apehex}
\newcommand{\contactlocation}{France, Terre}
\newcommand{\contactmail}{ape.hex@domain.com}
\newcommand{\contactname}{Ape Hex}
\newcommand{\contactphone}{+01 234-567-890}
\newcommand{\contactqualification}{Développeur Fullstack $\sim$ Ingénieur}

\newcommand{\urlgithubctf}{https://\contactgithub}
\newcommand{\urlgithubprofile}{https://\contactgithub}
\newcommand{\urlhuggingfaceprofile}{https://huggingface.co/apehex}
\newcommand{\urlhuggingfacearticlediffusion}{https://huggingface.co/apehex} % todo
\newcommand{\urlhuggingfacearticletokenization}{https://huggingface.co/blog/apehex/this-title-is-already-tokenized}
\newcommand{\urlmailaddress}{mailto:\contactmail}
\newcommand{\urlphonenumber}{tel:\contactphone}

%----------------------------------------------------------------------------------------
%   TITLE AND CONTACT INFORMATION
%----------------------------------------------------------------------------------------

\begin{minipage}[t]{0.5\textwidth}
    \vspace{-\baselineskip} % Required for vertically aligning minipages
    {\fontsize{16}{20} \textcolor{black}{\textbf{\MakeUppercase{\contactname}}}}\\\\
    {\Large \contactqualification}
\end{minipage}
\hfill
\begin{minipage}[t]{0.2\textwidth}
    \vspace{-\baselineskip}
    \icon{Phone}{11}{\href{\urlphonenumber}{\contactphone}}\\
    \icon{MapMarker}{11}{\contactlocation}\\
\end{minipage}
\begin{minipage}[t]{0.27\textwidth}
    \vspace{-\baselineskip}
    \icon{Envelope}{11}{\href{\urlmailaddress}{\contactmail}}\\
    \icon{Github}{11}{\href{\urlgithubprofile}{\contactgithub}}\\
\end{minipage}


%----------------------------------------------------------------------------------------
%   INTRODUCTION, SKILLS AND TECHNOLOGIES
%----------------------------------------------------------------------------------------

\begin{minipage}[t]{0.46\textwidth}
    \cvsect{Profile}\\
    Curieux, avec une attirance pour les défis techniques.
\end{minipage}
\hfill
\begin{minipage}[t]{0.465\textwidth}
    \cvsect{Objectif}\\
    Intégration de l'intelligence artificielle aux circuits.
\end{minipage}

%----------------------------------------------------------------------------------------
%   Projects
%----------------------------------------------------------------------------------------

% \cvsect{Projects}
% \begin{entrylist}
%     \entry
%         {Technology}
%         {My project 1}
%         {github.com link}
%         {%Dummy text
%         \lipsum[1][1-3]}
%     \entry
%         {Technology}
%         {My project 2}
%         {github.com link}
%         {%Dummy text
%         \lipsum[1][1-3]}
%     \entry
%         {Technology}
%         {My project 3}
%         {github.com link}
%         {%Dummy text
%         \lipsum[1][1-3]}
%     \entry
%         {Technology}
%         {My project 4}
%         {github.com link}
%         {%Dummy text
%         \lipsum[1][1-3]}
% \end{entrylist}

%----------------------------------------------------------------------------------------
%   EXPERIENCE
%----------------------------------------------------------------------------------------

\cvsect{Expérience}
\begin{entrylist}
\entry
    {1/2024 -- actuel}
    {\textbf{Chercheur en Apprentissage Machine}}
    {Indépendant}
    {\vspace{-10pt}
    \begin{itemize}[noitemsep,topsep=0pt,parsep=0pt,partopsep=0pt, leftmargin=-1pt]
        \item création des couches \textbf{\href{\urlhuggingfacearticletokenization}{composite embedding}} et \textbf{\href{\urlhuggingfacearticletokenization}{binary prediction}}, surpassant la tokenisation.
        \item développement de modèles binaires, égalant l'état de l'art de la (dé)compilation de code solidity.
        \item unification des modèles génératifs de language et d'image: \textbf{\href{\urlhuggingfacearticlediffusion}{byte level, latent diffusion of language}}.
    \end{itemize}}
\entry
    {1/2015 -- actuel}
    {\textbf{Contributeur Open Source}}
    {Indépendant}
    {\vspace{-10pt}
    \begin{itemize}[noitemsep,topsep=0pt,parsep=0pt,partopsep=0pt, leftmargin=-1pt]
        \item participation aux projets que j'utilise, dernièrement : \textbf{\href{\urlgithubprofile}{Keras, Transformers, Unsloth, Meta BLT}}, etc.
        \item développement et partage de divers \textbf{\href{\urlhuggingfaceprofile}{articles, modèles, datasets}}, \textbf{\href{\urlgithubprofile}{outils, patrons web et LaTeX}}.
    \end{itemize}}
\entry
    {7/2023 -- 1/2024}
    {\textbf{Chercheur en Sécurité Web3}}
    {Forta Network}
    {\vspace{-10pt}
    \begin{itemize}[noitemsep,topsep=0pt,parsep=0pt,partopsep=0pt, leftmargin=-1pt]
        \item état de l'art des techniques émergentes utilisées par des acteurs malveillants sur la blockchain.
        \item développement de bots pour inspecter en continu les transactions et détecter les tentatives de hack.
    \end{itemize}}
\entry
    {4/2023 -- 6/2023}
    {\textbf{Développeur Web3}}
    {DIAdata et Reserve-Protocol}
    {\vspace{-10pt}
    \begin{itemize}[noitemsep,topsep=0pt,parsep=0pt,partopsep=0pt, leftmargin=-1pt]
        \item intégration de Reserve-Protocol avec d'autres protocoles tout en garantissant la surcollatéralisation.
        \item développement d'un daemon pour surveiller la santé et la précision des oracles pour DIAdata.
    \end{itemize}}
\entry
    {3/2020 -- 12/2021}
    {\textbf{Bounty Hunter}}
    {Indépendant}
    {\vspace{-10pt}
    \begin{itemize}[noitemsep,topsep=0pt,parsep=0pt,partopsep=0pt, leftmargin=-1pt]
        \item identification et résolution de vulnérabilités dans diverses applications mobiles et sites web.
    \end{itemize}}
\entry
    {1/2017 -- 12/2022}
    {\textbf{Développeur Fullstack Web2}}
    {Indépendant}
    {\vspace{-10pt}
    \begin{itemize}[noitemsep,topsep=0pt,parsep=0pt,partopsep=0pt, leftmargin=-1pt]
        \item transformation de savoir d'experts en outils, liant calcul technique et gestion.
        \item modernisation d'applications existantes avec des interfaces web et optimisations par IA.
    \end{itemize}}
\entry
    {9/2013 -- 12/2016}
    {\textbf{Consultant}}
    {C3-Expert}
    {\vspace{-10pt}
    \begin{itemize}[noitemsep,topsep=0pt,parsep=0pt,partopsep=0pt, leftmargin=-1pt]
        \item réalisation d'études techniques et simulations pour des clients tels qu'Arcelor, Koniambo Nickel, etc.
        \item amélioration des outils, avec des calculs techniques plus précis et pragmatiques.
    \end{itemize}}
\entry
    {1/2011 -- 6/2011}
    {\textbf{Stage}}
    {Total}
    {\vspace{-10pt}
    \begin{itemize}[noitemsep,topsep=0pt,parsep=0pt,partopsep=0pt, leftmargin=-1pt]
        \item documentation, débogage et amélioration d'un outil d'optimisation des mélanges d'hydrocarbures.
    \end{itemize}}
\end{entrylist}

%----------------------------------------------------------------------------------------
%   EDUCATION
%----------------------------------------------------------------------------------------

\cvsect{Formation}
\begin{entrylist}
    \entry
        {1/2023 - 03/2023}
        {\textbf{Web3 Training}}
        {OpenZeppelin and CTFs}
        {Perfectionnement en participant à des compétitions comme Ethernaut et Damn-Vulnerable-DEFI.}
    \entry
        {3/2020 - 10/2021}
        {\textbf{Cybersecurity Training}}
        {Cybrary and Coursera}
        {Formation approfondie en cybersécurité, incluant des CTFs (\textbf{\href{\urlgithubctf}{Hackerbox, Pentesterlab, OverTheWire}}).}
    \entry
        {7/2015 - 08/2015}
        {\textbf{Formations en Apprentissage Machine}}
        {Coursera and Udemy}
        {Cours intensifs en apprentissage machine (Coursera, Udemy) et projets sur les modèles génératifs.}
    \entry
        {9/2008 - 10/2012}
        {\textbf{Diplôme d'Ingénieur}}
        {\'{E}cole Centrale de Lyon}
        {Spécialisation en business development et data mining.}
\end{entrylist}

%----------------------------------------------------------------------------------------
%   Skills & Hobbies
%----------------------------------------------------------------------------------------

\begin{minipage}[t]{0.48\textwidth}
    \cvsect{Compétences}
    \begin{entrylist}[]
    \entry
        {ML}
        {Tensorflow, Keras, Pytorch, etc.}
        {}
        {}
    \entry
        {Bureau}
        {LaTeX, LibreOffice, Excel, etc.}
        {}
        {}
    \entry
        {Science}
        {Octave, Matlab, SciPy, SymPy, etc.}
        {}
        {}
    \entry
        {Scripting}
        {Python, Node, Shell, etc.}
        {}
        {}
    \entry
        {Web}
        {HTML, CSS, JS, Jekyll, BurpSuite, etc.}
        {}
        {}
    % \entry
    %     {Web3}
    %     {Solidity, EVM assembly.}
    %     {}
    %     {}
    \end{entrylist}
\end{minipage}
\hfill
\begin{minipage}[t]{0.48\textwidth}
    \cvsect{Loisirs}
    \begin{entrylist}[]
    \entry
        {Jeux}
        {Jeux de société et jeux vidéos.}
        {}
        {}
    \entry
        {Science}
        {Informatique quantique, maths, SF.}
        {}
        {}
    \entry
        {Sports}
        {Escalade, alpinisme.}
        {}
        {}
    \entry
        {Tech}
        {\href{\urlgithubctf}{CTFs}, notamment forensics et cryptography.}
        {}
        {}
    \end{entrylist}
\end{minipage}

%----------------------------------------------------------------------------------------

\end{document}
